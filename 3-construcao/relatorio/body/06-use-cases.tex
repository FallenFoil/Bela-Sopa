% ---------------------------------------------------------------------------- %

\section{Modelação de \emph{Use Cases}}
\label{cap:use-cases}

Uma vez terminados os processos de modelação de domínio e de levantamento de requisitos, procedeu-se à modelação de \emph{use cases}. Neste capítulo são apresentados os resultados dessa fase.

% ---------------------------------------------------------------------------- %

\subsection{Identificação dos \emph{use cases}}
\label{sec:use-cases:identificacao}

Os use cases foram identificados através da análise dos requisitos de utilizador, de forma a satisfazer todas as funcionalidades desejadas. Deste modo, os use cases identificados podem corresponder a uma ou mais funcionalidades, oferecidas pelo sistema, visto que, algumas delas correspondem a cenários alternativos de outras.

Foram subentendidos e omitidos no diagrama de use cases, use cases acessórios relativos à conta do utilizador, tais como, \emph{logout} e \emph{editar informações de conta}, por já possuírem um comportamento normalizado. Apenas foi mencionado o use case \emph{Autenticar} pelo facto dos outros use cases possuírem como pré-condição a autentificarão do utilizador, no entanto, não é apresentada a especificação deste.

% ---------------------------------------------------------------------------- %

\subsection{Especificação dos \emph{use cases}}
\label{sec:use-cases:especificacao}

Apresentam-se nesta secção alguns exemplos representativos de especificações de \emph{use cases}. As especificações completas dos restantes \emph{use cases} identificados são incluídas no \refane{ane:use-cases-spec}.

% ---------------------------------------------------------------------------- %

\begin{table}[ht]
  \centering
  \tabelausecase
  \begin{tabularx}{\textwidth}{|>{\raggedright\let\newline\\\arraybackslash\hspace{0pt}}p{2.5cm}|>{\raggedright\let\newline\\\arraybackslash\hspace{0pt}}X|>{\raggedright\let\newline\\\arraybackslash\hspace{0pt}}X|}
    \hline
    \emph{Use case}: & \multicolumn{2}{l|}{Registar conta} \\ \hline
    Pré-condição: & \multicolumn{2}{l|}{Ator não existe} \\ \hline
    Pós-condição: & \multicolumn{2}{l|}{Ator adicionado ao sistema} \\ \hline
     & \textbf{Ator} & \textbf{Sistema} \\ \hline
    \multirow[t]{4}{=}{Comportamento Normal} &  & 1. Pergunta o email, nome, distrito e palavra-chave \\ \cline{2-3}
     & 2. Fornece dados &  \\ \cline{2-3}
     &  & 3. Valida dados \\ \cline{2-3}
     &  & 4. Adiciona utente \\ \hline
    \multirow[t]{2}{=}{Comportamento Alternativo 1 [Dados inválidos] (passo 3)} &  & 3.1. Indica ao cliente que os dados são inválidos. \\ \cline{2-3}
     &  & 3.2. Volta para 1 \\ \hline
\end{tabularx}
  \caption{Especificação do \emph{use case} ``registar conta''.}
  \label{tab:uc-registar-conta}
\end{table}

% ---------------------------------------------------------------------------- %

% ---------------------------------------------------------------------------- %

\begin{table}[ht]
  \centering
  \tabelausecase
  \begin{tabularx}{\textwidth}{|>{\raggedright\let\newline\\\arraybackslash\hspace{0pt}}p{2.5cm}|>{\raggedright\let\newline\\\arraybackslash\hspace{0pt}}X|>{\raggedright\let\newline\\\arraybackslash\hspace{0pt}}X|}
    \hline
    \emph{Use case}: & \multicolumn{2}{l|}{Procurar receitas} \\ \hline
    Pré-condição: & \multicolumn{2}{l|}{Estar autenticado} \\ \hline
    Pós-condição: & \multicolumn{2}{l|}{Encontrou pelo menos uma receita} \\ \hline
     & \textbf{Ator} & \textbf{Sistema} \\ \hline
    \multirow[t]{4}{=}{Comportamento Normal} & 1. Introduz nome de uma receita &  \\ \cline{2-3}
     &  & 2. Inicia pesquisa da receita pelo nome \\ \cline{2-3}
     &  & 3. Valida pesquisa \\ \cline{2-3}
     &  & 4. Mostra receitas encontradas \\ \hline
    \multirow[t]{3}{=}{Comportamento Alternativo 1 [Ator procura receitas por tag] (passo 1)} & 1.1. Introduz uma tag &  \\ \cline{2-3}
     &  & 1.2. Inicia pesquisa de receitas pela tag \\ \cline{2-3}
     &  & 1.3. Volta ao passo 3 \\ \hline
    Exceção 1 [Nenhuma receita corresponde à pesquisa] (passo 3) &  & 3.1. Indica que nenhuma receita foi encontrada \\ \hline
\end{tabularx}
  \caption{Especificação do \emph{use case} ``procurar receitas''.}
  \label{tab:uc-procurar-receitas}
\end{table}

% ---------------------------------------------------------------------------- %

% ---------------------------------------------------------------------------- %

\begin{table}[ht]
  \centering
  \tabelausecase
  \begin{tabularx}{\textwidth}{|>{\raggedright\let\newline\\\arraybackslash\hspace{0pt}}p{2.5cm}|>{\raggedright\let\newline\\\arraybackslash\hspace{0pt}}X|>{\raggedright\let\newline\\\arraybackslash\hspace{0pt}}X|}
    \hline
    \emph{Use case}: & \multicolumn{2}{l|}{Confecionar receita} \\ \hline
    Pré-condição: & \multicolumn{2}{l|}{Estar autenticado} \\ \hline
    Pós-condição: & \multicolumn{2}{l|}{Receita confecionada} \\ \hline
     & \textbf{Ator} & \textbf{Sistema} \\ \hline
    \multirow[t]{14}{=}{Comportamento Normal} &  & 1. Pergunta se possui todos os ingredientes. \\ \cline{2-3}
     & 2. Confirma &  \\ \cline{2-3}
     &  & 3. Pergunta se possui todos os utencílios \\ \cline{2-3}
     & 4. Confirma &  \\ \cline{2-3}
     &  & 5. Recebe o primeiro processo \\ \cline{2-3}
     &  & 6. Mostra o processo atual, as tarefas neste e os ingredientes, utensílios e técnicas nas respetivas tarefas \\ \cline{2-3}
     & 7. Confirma a execução de todas as tarefas &  \\ \cline{2-3}
     & 8. Avança &  \\ \cline{2-3}
     &  & 9. Verifica próximo processo \\ \cline{2-3}
     &  & 10. Não existe mais processos \\ \cline{2-3}
     &  & 11. Guarda a conclusão da receita \\ \cline{2-3}
     &  & 12. Pergunta a opinião sobre a conceção da receita \\ \cline{2-3}
     & 13. Responde &  \\ \cline{2-3}
     &  & 14. Guarda resposta \\ \hline
    Comportamento Alternativo 1 [Existe próximo processo] (passo 10) &  & 10.1. Recebe o próximo processo e volta ao passo 6 \\ \hline
    Comportamento Alternativo 2 [Utilizador retrocede o processo] (passo 7) &  & 7.1. Recebe o processo anterior e volta ao passo 6 \\ \hline
    \multirow[t]{3}{=}{Comportamento Alternativo 3 [Processo anterior não existe] (passo 8.1)} &  & 8.1.1. Indica que não existe um processo anterior \\ \cline{2-3}
     &  & 8.1.2. Mantêm o processo atual \\ \cline{2-3}
     &  & 8.1.3. Volta ao passo 6 \\ \hline
    Comportamento Alternativo 4 [Ator não responde] (passo 13) &  & 13.1. Termina Confecionar receita \\ \hline
    \multirow[t]{2}{=}{Exceção 1 [Não confirma] (passo 2 ou 4)} &  & 2.1. Indica para voltar quando possuir tudo \\ \cline{2-3}
     &  & 2.2. Sai do serviço Confecionar receita \\ \hline
    Exceção 2 [Utilizador cancela Confecionar receita] (passo indeterminado) &  & 2.1. Sai do serviço Confecionar receita \\ \hline
\end{tabularx}
  \caption{Especificação do \emph{use case} ``confecionar receita''.}
  \label{tab:uc-confecionar-receita}
\end{table}

% ---------------------------------------------------------------------------- %


% ---------------------------------------------------------------------------- %

%  Cliente/Administrador
% - Autenticar
% - Sair
% - Consultar/Editar informação da conta.
% - Registar conta
% \guilherme{Registar conta deveria se global visto que os dados são os mesmos} - Registar conta de administrador
% - Apagar Conta.
% - Remover Conta

% - Procurar receitas
% - Consultar receita
% - Adiciona/remove receita aos favoritos.

% - Iniciar Confeção da receita
% % Acho que isto tudo faz parte do confecionar receita.
% %- Avançar/Retroceder nos processos
% %- Iniciar temporizador
% %- Parar temporizador
% %- Concluir Confeção
% %- Cancelar Confeção da receita

% - Gerar ementa semanal
% - Editar ementa semanal
% - Adicionar/remover receita à ementa semanal.
% - Gerar lista de ingredientes

% - Aceder a serviços externos (ao domicilio).

% ---------------------------------------------------------------------------- %

% Luís:
%     -Aceder a serviçoes externos;
%     -Adicionar receita favoritos;
%     -Remover receita favoritos.

% Fábio:
%     -Gerar ementa semanal;
%     -Editar ementa semanal;
%     -Registar conta administrador;

% César:
%     -Gerar lista ingredientes;
%     -Procurar receitas;
%     -Consultar receita.

% Guilherme:

%     - Consultar/Editar informação da conta.
%     - Registar conta de cliente
%     - Registar conta de administrador
%     - Apagar Conta.

% ---------------------------------------------------------------------------- %
