% ---------------------------------------------------------------------------- %

\section{Fundamentação do Sistema}
\label{cap:fundamentacao}

Tendo-se apresentado o caso de estudo e identificado as motivações e objetivos para o desenvolvimento do sistema em questão, este capítulo define agora a sua identidade e fundamenta a sua construção tendo em conta a viabilidade e utilidade do mesmo.

% ---------------------------------------------------------------------------- %

\subsection{Identidade do sistema}
\label{sec:fundamentacao:identidade}

O \emph{Bela Sopa} consiste num assistente pessoal que tem como objetivo acompanhar o utilizador durante todo o processo de confeção de uma receita de sopa, destinando-se à faixa etária adulta e não sendo direcionado a menores de idade.

Durante a confeção da receita, o assistente é capaz de interagir com o utilizador, designando quais os ingredientes necessários e os passos serem a executados, proporcionando também informação relativa aos utensílios e técnicas a utilizar durante a produção do prato, sendo assim capaz de responder a diferentes cenários alternativos.

Várias caraterísticas que definem a identidade do sistema são apresentadas na \reftab{tab:fundamentacao:identidade}.

\begin{table}[ht]
  \centering
  \begin{tabular}{ | l | l | }
    \hline
    \textbf{Nome} & \emph{Bela Sopa} \\ \hline
    \textbf{Categoria} & Assistente pessoal \\ \hline
    \textbf{Designação} & Assistente pessoal de cozinha para a confeção de sopas \\ \hline
    \textbf{Idioma} & Português \\ \hline
    \textbf{Faixa etária} & Adultos \\ \hline
    \textbf{Características} & \emph{User friendly}, personalizável e prático\\ \hline
    \textbf{Empresa cliente} & \emph{Gota Doce} \\ \hline
  \end{tabular}
  \caption{Ficha de projeto.}
  \label{tab:fundamentacao:identidade}
\end{table}

% ---------------------------------------------------------------------------- %

\subsection{Justificação, viabilidade e utilidade do sistema}
\label{cap:fundamentacao:justificacao}

Justifica-se a realização deste projeto para substituir o serviço \emph{Escola de Cozinha} por um sistema interativo, eficiente, inteligente e de melhor qualidade. Com este projeto a nossa empresa obtém um cliente de grande escala enquanto que o cliente obtém um serviço inovador.

Antes do nosso cliente ter investido neste projeto, teve que ter um contexto para o mesmo. Para isso, este teve de investigar os seus clientes atuais e desejados para entender a vida dos mesmos e as suas necessidades, ficando com uma ideia do futuro do projeto para atender às demandas dos seus clientes.

Isto foi feito através de um questionário, o qual se concluiu que cerca de 80\% das pessoas, tanto clientes atuais como possíveis futuros clientes, procuram respostas tecnológicas aos seus problemas diários, e que 70\% destes utilizariam um assistente pessoal para ajudar na sua culinária, enquanto que apenas 8\% possuem elevada experiência culinária. 

Este projeto é viável pelas seguintes razões:

\begin{itemize}

    \item Não existem obstáculos legais (leis ou patentes) que proíbam ou limitem o desenvolvimento e comercialização do produto resultante;
    
    \item A nossa equipa prevê que o desenvolvimento deste produto é exequível com os recursos tecnológicos atuais;
    
    \item É geograficamente favorável devido ao elevado número de super e hipermercados e à sua dispersão pelo território nacional;
    
    \item É vantajoso em termos de visibilidade pública, uma vez que a cadeia \emph{Gota Doce} está associada ao produto;
    
    \item Não existe impactos ambientais associados ao produto;
    
    \item Financeiramente, prevê-se que o projeto seja rentável a curto prazo.
    
\end{itemize}

% ---------------------------------------------------------------------------- %
