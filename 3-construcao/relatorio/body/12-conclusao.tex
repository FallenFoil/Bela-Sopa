% ---------------------------------------------------------------------------- %

\section{Conclusões e Trabalho Futuro}
\label{cap:conclusao}

% Neste relatório apresentaram-se as fases de fundamentação e especificação do assistente pessoal de cozinha \emph{Bela Sopa}, cujo desenvolvimento foi encomendado pela cadeia de supermercados e hipermercados \emph{Gota Doce}. O sistema é baseado no serviço \emph{online} \ESCOLADECOZINHA, detido pela mesma empresa.

Neste relatório apresentou-se o trabalho prático realizado no âmbito da Unidade Curricular de Laboratórios de Informática IV, do curso de Mestrado Integrado em Engenharia Informática da Universidade do Minho, no ano letivo de 2018/2019. O caso de estudo considerado centra-se no desenvolvimento do assistente pessoal de cozinha \emph{Bela Sopa}, encomendado pela cadeia de supermercados e hipermercados \emph{Gota Doce}.

Foi primeiramente fundamentada a construção do sistema tendo em conta a sua utilidade e viabilidade, tendo-se também elaborado um plano para o seu desenvolvimento. Verificado-se que a construção do sistema seria vantajosa e que o seu processo de desenvolvimento cumpriria o orçamento e prazos estabelecidos, procedeu-se à fase de especificação do mesmo.

O trabalho realizado nessa mesma fase foi em seguida pormenorizado, apresentando-se com particular detalhe os resultados dos processos de modelação de domínio, levantamento e análise de requisitos, modelação de \emph{use cases}, prototipagem da interface de utilizador e modelação da arquitetura interna do sistema.

Por fim, descreveu-se o processo de construção do sistema e apresentou-se o produto resultante, enumerando-se também as tecnologias nas quais este se baseia e especificando-se o procedimento de instalação do sistema.

Devido a restrições de tempo, algumas funcionalidades delineadas na fase de especificação do sistema não foram concretizadas aquando da sua construção. Em particular, embora as receitas confecionadas e respetiva avaliação e dificuldade reportadas pelo utilizador sejam armazenadas e exibidas no seu histórico pessoal, esta informação não é utilizada para filtrar ou organizar a lista de receitas do sistema apresentada ao mesmo. Adicionalmente, a funcionalidade de visualização de lojas próximas poderá ser estendida com a capacidade de indicar trajetos até uma determinada loja. A implementação destas funcionalidades omissas constitui trabalho futuro.

Provando-se a adequação do sistema construído e o cumprimento dos seus objetivos, será considerado o alargamento do seu âmbito a formas de culinária além da confeção de sopas. Esta evolução é também identificada como trabalho futuro.

% Em particular, caso o utilizador declare que não tem todos os ingredientes necessários a uma determinada receita, o sistema não fornece diretamente informação sobre os locais onde os pode adquirir e os respetivos preços, embora o utilizador possa determinar a localização.

% ---------------------------------------------------------------------------- %
