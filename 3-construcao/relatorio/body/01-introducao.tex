% ---------------------------------------------------------------------------- %

\section{Introdução}
\label{cap:introducao}
\pagenumbering{arabic}

Este relatório documenta o trabalho desenvolvido no âmbito da Unidade Curricular de Laboratórios de Informática IV, do curso de Mestrado Integrado em Engenharia Informática da Universidade do Minho, no ano letivo de 2018/2019.

O caso de estudo considerado centra-se no desenvolvimento do assistente pessoal de cozinha \emph{Bela Sopa}, sucessor ao serviço \emph{online} intitulado \emph{Escola de Cozinha} disponibilizado pela cadeia de supermercados e hipermercados portuguesa \emph{Gota Doce}. Este capítulo contextualiza e apresenta o caso de estudo, descrevendo também as motivações e os objetivos do projeto, e delineia a estrutura do presente relatório.

% ---------------------------------------------------------------------------- %

\subsection{Contextualização e caso de estudo}
\label{sec:introducao:contextualizacao}

O \emph{Gota Doce} é uma cadeia de supermercados e hipermercados portuguesa, sediada em Lisboa. Foi fundada em 1983 pelo grupo empresarial \emph{Merónimo Jartins} em colaboração com a empresa belga \emph{Gelhaize Droup}. A cadeia possui atualmente mais de 400 lojas físicas distribuídas por cerca de 290 localidades portuguesas, contando com aproximadamente 32 milhares de colaboradores e 700 mil visitas diárias.

Além desta vasta rede de pontos de serviço, a empresa \emph{Gota Doce} oferece também vários serviços \emph{online}, destacando-se o serviço de entrega de produtos ao domicílio. Outros serviços têm como objetivo a fidelização de clientes e a manutenção da imagem da marca, como é o caso da \emph{Escola de Cozinha}, uma plataforma \emph{online} que disponibiliza informação relacionada com culinária. Esta oferece auxílio à confeção de refeições por parte dos seus utilizadores, fornecendo principalmente descrições estáticas de receitas, técnicas de cozinha e ingredientes.

Sendo este serviço utilizado por uma quantidade considerável de clientes da cadeia \emph{Gota Doce}, a empresa pretende agora investir na melhoria da plataforma, focando a possibilidade de adaptação e personalização do serviço a cada um dos seus utilizadores. Com esse objetivo, foi pedido o desenvolvimento de um assistente pessoal de cozinha que substituirá o atual serviço \emph{Escola de Cozinha}. Por forma a se provar a viabilidade e adequação desta nova plataforma, a sua versão inicial será dedicada exclusivamente à confeção de sopas. O caso de estudo no qual o trabalho documentado neste relatório se baseia corresponde então ao desenvolvimento do assistente em causa, intitulado \emph{Bela Sopa}.

% Para além desta vasta disponibilidade de pontos de serviço, com as várias lojas apresentadas em diversos pontos do país, o \emph{Gota Doce} também oferece os seus serviços \emph{online} através do seu \emph{website}. Os clientes podem assim usufruir do serviço de compras \emph{online}, o qual permite o acesso aos seus produtos de uma forma mais abrangente e conveniente.

% Embora o \emph{website} sirva como uma extensão do alcance dos seus serviços de venda de produtos, o \emph{Gota Doce} decidiu fornecer outros serviços, entre os quais se encontra presente a plataforma \emph{Escola de Cozinha}.

% A \emph{Escola de Cozinha} é uma plataforma que procura transmitir conhecimento que ajuda os utilizadores da mesma a confecionar variadas refeições, a aumentar o seu domínio culinário e a incentivar a confeção de refeições caseiras.

% De acordo com as características dos seus conteúdos, a \emph{Escola de Cozinha} divide-os em 5 secções: técnicas, ingredientes, vídeos, receitas e histórias de cozinha.

% \itemizedpar{Técnicas.}

% Nesta secção são apresentadas receitas nas quais são introduzidas técnicas que requerem alguma experiência, ou que merecem ser realçadas. Para facilitar a aprendizagem destas técnicas estas são apresentadas em conjunto com imagens que ilustram detalhadamente os passos a tomar em cada estágio.

% \itemizedpar{Ingredientes.}

% Nesta secção diversos ingredientes são descritos em detalhe, desta forma o utilizador poderá conhecer melhor o ingrediente e desta forma conseguir incluir o mesmo nas sua refeições.

% \itemizedpar{Vídeos.}

% Contém diversos vídeos nos quais são demonstradas técnicas e confecionadas receitas, desta forma os utilizadores têm uma maior facilidade na compreensão das técnicas e passos presentes nas receitas.   

% \itemizedpar{Receitas.}

% São apresentadas diversas receitas, cada uma caraterizada por dificuldade (fácil, média, difícil) e tempo de confeção, tipo de prato (entrada, sobremesa, etc) e número de porções, isto é, para quantas pessoas é a receita. Para além dessa caraterização, é apresentada uma descrição do prato e são também descritos os procedimentos passo a passo, para confecionar o prato, conjuntamente com a lista de ingredientes e medidas dos mesmos, tabela de valores nutricionais.

% \itemizedpar{Histórias de cozinha.} 

% Constitui um conjunto de artigos que fornecem um complemento informativo na cultura culinária do utilizador. Os assuntos abordados englobam as componentes estéticas e técnicas da cozinha, fornecendo um contributo para as habilidades culinárias do utilizador.

% A \emph{Escola de Cozinha} tem registado um grande volume de visitas, desde a sua introdução e em analise verificou-se um aumento do número de clientes e produtos adquiridos através da loja online do \emph{Gota Doce}. Devido a este sucesso foi decidido o investimento na melhoria da \emph{Escola de Cozinha} para continuar a explorar essa possibilidade de aumento de negócio.

% Com essa ideia em mente, o \emph{Gota Doce} abordou-nos e após discussão de ideias surgiu a aplicação \emph{Bela Sopa}, um assistente pessoal de cozinha, através de uma tentativa de adaptar e personalizar o serviço a cada um dos utilizadores.

% Como fase inicial o assistente pessoal de cozinha apenas irá ajudar à confeção de sopas de forma a estudar melhor a receção desta nova aplicação por parte do público, sem ser necessário um orçamento elevado. Através desta limitação do âmbito da aplicação é possível a ampliação das funcionalidades da aplicação para que a quando a migração dos restantes conteúdos a aplicação já se apresente a um nível de qualidade elevado.

% ---------------------------------------------------------------------------- %

\subsection{Motivação e objetivos}
\label{sec:introducao:motivacao-objetivos}

Embora não se pretenda monetizar diretamente a plataforma em questão (\emph{e.g.}, através de subscrições pagas para acesso aos serviços por esta disponibilizados), objetiva-se com a sua construção angariar e fidelizar clientes para os principais serviços oferecidos pela empresa \emph{Gota Doce}. A título de exemplo, o sistema poderá promover a utilização desses outros serviços ao indicar que os ingredientes utilizados por uma determinada receita podem ser obtidos em lojas físicas \emph{Gota Doce} próximas ou através do serviço de entrega ao domicílio disponibilizado pela empresa. A plataforma poderá também aumentar a exposição dos clientes da cadeia a folhetos promocionais e outros materiais publicitários.

De forma geral e sumária, pretende-se que a construção da plataforma em questão contribua para o crescimento e manutenção do volume de negócio da empresa \emph{Gota Doce}. Estes objetivos serão detalhados e quantificados em secção posterior deste relatório.

% O cliente decidiu iniciar este projeto com o intuito de criar uma relação mais íntima com os seus clientes, de suavizar o acesso às suas receitas no seu \emph{website} e promover os seus produtos e o serviço de entrega ao domicílio. Com este projeto espera-se aumentar o numero de compras de produtos da marca \emph{Gota Doce}, o reconhecimento e divulgação da marca, o que leva a uma maior atração de clientes e consequentemente um maior lucro.

% ---------------------------------------------------------------------------- %

\subsection{Estrutura do relatório}
\label{sec:introducao:estrutura-relatorio}

Este documento apresenta a seguinte estrutura:

\begin{itemize}

  \item No \refcap{cap:fundamentacao} fundamenta-se o sistema, tendo em conta a sua utilidade e viabilidade e justificando-se o seu desenvolvimento;
  
  \item No \refcap{cap:planeamento} é apresentado o planeamento do projeto, juntamente com um modelo inicial do sistema, recursos necessários à concretização do mesmo, medidas de sucesso e um plano detalhado do seu desenvolvimento;
  
  \item No \refcap{cap:dominio} inicia-se a fase de especificação do sistema descrevendo-se o processo de modelação de domínio e os respetivos resultados;
  
  \item No \refcap{cap:requisitos} apresenta-se o levantamento e análise de requisitos, enumerando-se os requisitos identificados e efetuando-se uma análise geral dos mesmos;
  
  \item No \refcap{cap:use-cases} procede-se à modelação de \emph{use cases}, começando-se por identificar a totalidade dos \emph{use cases} considerados e apresentando-se depois a sua especificação;
  
  \item No \refcap{cap:interface} é executada a prototipagem da interface de utilizador do sistema, recorrendo-se primariamente a esquemas do seu aspeto gráfico;
  
  \item No \refcap{cap:arquitetura} apresenta-se e especifica-se de forma holística a arquitetura interna do sistema;
  
  \item No \refcap{cap:dados} detalha-se a especificação da camada de dados do sistema, modelando-se também a base de dados subjacente;
  
  \item No \refcap{cap:negocio} especifica-se da camada de negócio do sistema recorrendo primariamente a diagramas de classes e de sequência;
  
  \item No \refcap{cap:construcao} descreve-se a fase de construção do sistema, apresentando-se também o produto resultante;
  
  \item No \refcap{cap:conclusao} o relatório é concluído com um resumo do processo de desenvolvimento do sistema e com a identificação de trabalho futuro.

\end{itemize}

Adicionalmente, são incluídos dois anexos:

\begin{itemize}

  \item No \refane{ane:use-cases-spec} são reproduzidas as especificações de todos os \emph{use cases} identificados no \refcap{cap:use-cases};

  % \item No \refane{ane:dados-sbd-concetual} são descritas as entidades, relacionamentos e atributos do modelo concetual do sistema de bases de dados descrito na \refsec{sec:dados:sbd:concetual};

  % \item No \refane{ane:dados-diag-seq} são incluídos todos os diagramas de sequências desenvolvidos relativos à camada de dados descrita no \refcap{cap:dados};

  \item No \refane{ane:negocio-diag-seq} são incluídos todos os diagramas de sequência relativos à camada de negócio descrita no \refcap{cap:negocio}.

\end{itemize}
  
% ---------------------------------------------------------------------------- %
