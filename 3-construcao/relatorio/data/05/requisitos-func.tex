% ---------------------------------------------------------------------------- %

% REQUISITOS POR ADAPTAR:

    %NAO POR \defreqsys{reqs:estado-receita}{}{O estado atual da confeção da receita é guardado caso o utilizador saia do sistema, não cancelando a confeção da receita}
 
    %DONE \defreqsys{}{}{O sistema deve apresentar e explicar o “cozinhado” ou sugerir-lhe alguns vídeos de ajuda ou sites com informação pertinente (caso o utilizador não saiba como realizar o seu cozinhado);}
    
    %DONE \defreqsys{}{}{Em fase de preparação, o sistema deve estar preparado para redirecionar o utilizador para uma das lojas mais próximas referentes ao mesmo com todos os ingredientes necessários para a confeção da sopa ou disponibilizar o serviço de entrega ao domicilio(O utilizador pode alterar estes ingredientes posteriormente).}
    
    %DONE \defreqsys{}{}{Sistema deve ser capaz de recomeçar a explicação do processo, ou voltar a explicar um processo prévio caso preciso.}
    
    %DONE \defreqsys{}{}{Um processo deverá ser capaz de conter um temporizador para a fase em questão.}
    
    %NAO POR \defreqsys{}{}{Inicialmente o sistema deve apresentar os receitas com o maior número de consultas e as receitas com a melhor classificação.}
    
% MAIS REQUISITOS POR ADAPTAR:

    %NAO POR \item Inicialmente o sistema deve apresentar os receitas mais consultadas e as receitas com a melhor classificação.
    %DONE \item Depois de uma encomenda feita o sistema abre uma nova janela.
    %DONE \item Depois de um pedido de entrega ao domicilio o sistema redireciona o cliente para a janela de preparação da sopa.
    %DONE \item Permite a um utilizador fazer login através de email e password.
    %DONE \item Permite a um utilizador se registar, fornecendo o email, nome e password e distrito.
    %DONE \item Na ementa semanal são permitidas x sopas por refeição, 2 refeições por dia, 7 dias por semana.
    %DONE \item Deve ser capaz de apresentar as configurações relativas á conta do cliente, o seu perfil, histórico de receitas já realizadas e receitas favoritas. Deve mostrar estas receitas por nome ou etiqueta.
    %DONE \item O sistema deve permitir que o utilizador mude qualquer dado da sua respectiva conta.
    
    
    
    %DONE \item Possuir um interface bastante amigável,suportado por um gestor de diálogos sofisticados e capaz de sustentar uma “conversa” razoável com o utilizador ao longo de todos os processos de trabalho, sendo capaz de especificar o passo em que se encontra e quais os passos realizados.
    %DONE\item O sistema deve permitir aos seus utilizadores definir uma configuração inicial para o sistema, deixando-os escolher aquilo que usualmente pretendem cozinhar e o tipo de ingredientes que querem usar;
    %DONE     \begin{itemize}
    %DONE         \item É possível pré-definir o que se pretende cozinhar por etiquetas.
    %DONE     \end{itemize}
    %DONE \item O sistema deve apresentar os seus serviços e funcionalidades, explicando previamente o seu modo e princípios de funcionamento;
    %DONE \item O sistema permite recordar quais as receitas favoritas de um utilizador, sendo possível adicionar ou remover desta lista nas seguintes circunstancias:
    %DONE     \begin{enumerate}
    %DONE         \item No final da confeção de uma receita;
    %DONE         \item Na vista geral de uma receita.
    %DONE     \end{enumerate}
    %DONE \item As unidades utilizadas nos ingredientes da sopa são medidos com valores concretos e em alguns casos outras unidades equivalentes.

% ---------------------------------------------------------------------------- %

\begin{requisitos}{Tipos de utilizador, contas de utilizador e autenticação}

  \begin{requsr}{}{O sistema deverá suportar dois tipos de utilizador: (1) \emph{clientes} --- o público-alvo do sistema --- e (2) \emph{administradores} --- utilizadores responsáveis pela gestão da informação disponibilizada pelo sistema.}
  
    \reqsys{}{A cada utilizador é atribuída apenas uma \emph{conta de utilizador}.}
    
    \reqsys{}{Cada conta de utilizador é classificada como \emph{de administrador} ou \emph{de cliente}, correspondendo ao tipo de utilizador homónimo, sendo identificada por um nome de utilizador e salvaguardada por meio de uma palavra-chave.}
    
    \reqsys{}{Para utilizar o sistema, um utilizador deverá primeiro autenticar-se no mesmo indicando o nome de utilizador e a palavra-chave da sua conta.}
  
    \reqsys{}{Deve sempre existir, no mínimo, uma conta de administrador.}
  
    \reqsys{}{Um utilizador não autenticado pode criar contas de cliente, fornecendo um nome de utilizador, uma palavra-chave e, opcionalmente, um email.}
    
    \reqsys{}{Um utilizador autenticado pode revogar a sua autenticação.}
  
    \reqsys{}{Um administrador autenticado pode criar e eliminar contas de administrador.}
  
    \reqsys{}{Um administrador autenticado pode eliminar contas de cliente.}
  
    \reqsys{}{Um cliente autenticado pode eliminar a sua própria conta.}
  
  \end{requsr}
  

  \begin{requsr}{}{Ao administrador é permitido a adição ou remoção de receitas do sistema.}
    
    \reqsys{}{O administrador pode adicionar uma receita nova ao sistema, a qualquer momento, fornecendo todos os elementos constituintes da mesma.}
    
    \reqsys{}{O administrador pode remover qualquer receita existente no sistema, bastando para tal especificar o identificador da mesma.}
    
    \reqsys{}{Não é permitida a existência de duas receitas com o mesmo identificador}
    
  \end{requsr}
  
  \begin{requsr}{}{O utilizador pode alterar/remover informações da sua conta.}
    
    \reqsys{}{Toda a informação armazenada em associação à conta de um utilizador, pode ser modificada pelo mesmo, excetuando informações que seja utilizadas para efeitos de identificação da conta.}
    
    \reqsys{}{Quaisquer informações opcionais, introduzidas pelo utilizador, podem ser removidas pelo mesmo}
    
  \end{requsr}

\end{requisitos}

% ---------------------------------------------------------------------------- %

\begin{requisitos}{Requisitos de execução da confeção}

  \begin{requsr}{}{No inicio da confeção, o utilizador pode confirmar que possui todos os ingredientes para a receita.}
  
    \reqsys{}{O sistema deverá apresentar, em cada receita, uma breve descrição, imagens, número de doses, quais os utensílios e ingredientes necessários e as suas respetivas quantidades, passos da receita, tempo de confeção, dificuldade, etiquetas e uma breve informação nutricional.}
    
    \reqsys{}{Caso o utilizador não tenha todos os ingredientes, este pode ser redirecionado para um serviço em que seja possível a compra / encomenda dos ingredientes.}
    
  \end{requsr}

  \begin{requsr}{}{O utilizador pode recomeçar um processo quando quiser, pode ainda voltar para um processo anterior ou avançar para o próximo, caso exista.}
  
    \reqsys{}{Cada processo deve possuir uma ou mais tarefas que são independentes entre elas.}
    
    \reqsys{}{Só é permitido avançar para o próximo processo quanto todas as tarefas, do processo atual, forem concluídas, caso não haja mais processos a execução da receita termina.}
    
    \reqsys{}{Sistema deve ser capaz de recomeçar a explicação do processo atual, ou voltar a explicar um processo prévio caso seja requerido.}
    
  \end{requsr}

  \begin{requsr}{}{Em qualquer processo da confeção da receita, em caso de dúvida o utilizador pode pedir informações ao sistema, tais como: (1) o utilizador pode pedir informação extra das técnicas de utensílio e/ou materiais no processo especifico; (2) clarificar certos termos, mostrando sinónimos a um termo; (3) quanto tempo demora o processo.}
  
    \reqsys{}{O sistema deve apresentar e explicar a receita e, caso o utilizador não saiba como concluir alguma tarefa, sugerir alguns vídeos de ajuda ou \emph{websites} com informação pertinente.}
  
    \reqsys{}{O sistema deve mostrar a lista de ingredientes, utensílios e técnicas de cozinha que são utilizados na execução da receita.}
    
    \reqsys{}{As unidades utilizadas nos ingredientes são medidos com valores concretos e em alguns casos outras unidades equivalentes.}
    
    \reqsys{}{O sistema pode mostrar a duração que cada tarefa tem, em média, e consequentemente, pode também mostrar a duração do processo.}
    
    \reqsys{}{O sistema deve permitir a alteração das palavras técnicas por outras fornecidas pelo utilizador para simplificar os processos.}
    
    \reqsys{}{Em fase de preparação, o sistema deve estar preparado para redirecionar o utilizador para uma das lojas mais próximas referentes ao mesmo com todos os ingredientes necessários para a confeção da sopa ou disponibilizar o serviço de entrega ao domicilio (O utilizador pode alterar estes ingredientes posteriormente).}
    
  \end{requsr}

  \begin{requsr}{}{Durante a confeção da sopa, em caso de uma tarefa com tempo de espera, este pode ativar um temporizador do sistema.}
  
    \reqsys{}{O sistema alerta o utilizador sonoramente quando o temporizador terminar.}
    
    \reqsys{}{O temporizador deve ser apresentado em tempo real, mostrando os minutos e segundos restantes.}
    
  \end{requsr}

  \begin{requsr}{}{Após a confeção da receita, o utilizador pode classificar a dificuldade que teve na execução da mesma.}
  
    \reqsys{}{A receita confecionada é guardada num histórico.}
    
  \end{requsr}

  \begin{requsr}{}{O utilizador pode cancelar a confeção em qualquer parte da realização da mesma.}
  
    \reqsys{}{Ao cancelar a confeção é perdida qualquer informação relativa a esta.}
    
  \end{requsr}
  


\end{requisitos}

% ---------------------------------------------------------------------------- %

\begin{requisitos}{Outros}
  
    \begin{requsr}{}{O utilizador pode questionar o sistema relativamente às sua funcionalidades.}
    
        \reqsys{}{ O sistema deve apresentar os seus serviços e funcionalidades, explicando previamente o seu modo e princípios de funcionamento}
    
    \end{requsr}

  \begin{requsr}{}{O utilizador deve conseguir procurar receitas, técnicas e utensílios através do motor de busca (permitindo um acesso rápido ao que o utilizador deseja)}
  
    \reqsys{}{As receitas podem ser pesquisadas por nome ou etiqueta.}
    
    \reqsys{}{Os utensílios podem ser pesquisados por nome.}
    
    \reqsys{}{As técnicas podem ser pesquisadas por nome.}
    
  \end{requsr}

  \begin{requsr}{}{Quando requerido, o sistema deve anotar uma ementa semanal e preparar a lista dos ingredientes necessários para cada refeição, gerando quando necessário uma lista de compras geral para a semana}
  
    \reqsys{}{A ementa semanal pode ser gerada na sua totalidade no seu menu adequado.}
    
    \reqsys{}{No menu de uma receita, esta pode ser adicionada à ementa semanal nos dias e refeições escolhidas.}
    
    \reqsys{}{Só se pode adicionar uma receita por refeição, 2 refeições por dia, 7 dias por semana}
    
  \end{requsr}

  \begin{requsr}{}{Visualizar num conjunto de \emph{dashboards} (painéis de controlo) específico a um conjunto de dados relativos aos cozinhados realizados, os tempos de preparação, as dificuldades encontradas, os ingredientes utilizados, etc..}
  
    %\reqsys{}{TODO}
    
  \end{requsr}
  
  \begin{requsr}{}{O utilizador pode definir uma configuração inicial para o sistema, deixando-os escolher aquilo que usualmente pretendem cozinhar e o tipo de ingredientes que querem usar.}
  
    \reqsys{}{É possível pré-definir o que se pretende cozinhar por etiquetas}
    
    \reqsys{}{A configuração inicial pode ser editada mais tarde nas definições de conta}
    
  \end{requsr}

  \begin{requsr}{}{O utilizador pode adicionar ou remover uma receita da sua lista de favoritos.}
  
    \reqsys{}{As receitas encontradas pelos filtros usados na janela de pesquisa podem ser adicionadas da lista de favoritos, sendo possível a sua remoção da mesma.}
    
    \reqsys{}{No final da confeção de uma receita, o utilizador pode adicioná-la à lista de favoritos, sendo possível, posteriormente, a sua remoção da lista.}
    
  \end{requsr}
  
  \begin{requsr}{}{O utilizador pode definir uma configuração inicial para o sistema, indicando que tipo de receita pretende, usualmente, confecionar e o quais os ingredientes que intendem utilizar.}
  \end{requsr}
  
\end{requisitos}

% ---------------------------------------------------------------------------- %
