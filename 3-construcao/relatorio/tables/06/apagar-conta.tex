% ---------------------------------------------------------------------------- %

\begin{table}[ht]
  \centering
  \tabelausecase
  \begin{tabularx}{\textwidth}{|>{\raggedright\let\newline\\\arraybackslash\hspace{0pt}}p{2.5cm}|>{\raggedright\let\newline\\\arraybackslash\hspace{0pt}}X|>{\raggedright\let\newline\\\arraybackslash\hspace{0pt}}X|}
    \hline
    \emph{Use case}: & \multicolumn{2}{l|}{Apagar conta} \\ \hline
    Pré-condição: & \multicolumn{2}{l|}{Estar autenticado} \\ \hline
    Pós-condição: & \multicolumn{2}{l|}{Conta eliminada} \\ \hline
     & \textbf{Ator} & \textbf{Sistema} \\ \hline
    \multirow[t]{3}{=}{Comportamento Normal} & 1. Fornece o email da conta que pretende ser apaga. &  \\ \cline{2-3}
     &  & 2. Valida email \\ \cline{2-3}
     &  & 3. Elimina conta \\ \hline
    \multirow[t]{2}{=}{Comportamento Alternativo 1 [Ator é cliente] (passo 2)} &  & 2.1. Valida se a conta é do próprio \\ \cline{2-3}
     &  & 2.2. Elimina conta \\ \hline
    Exceção 1 [Email de cliente inválido] (passo 2) &  & 2.1. Informa que o email é inválido \\ \hline
    Exceção 2 [Dados inválidos] (passo 2) &  & 2.1. Informa que o email é invalido \\ \hline
\end{tabularx}
  \caption{Especificação do \emph{use case} ``apagar conta''.}
  \label{tab:uc-apagar-conta}
\end{table}

% ---------------------------------------------------------------------------- %
