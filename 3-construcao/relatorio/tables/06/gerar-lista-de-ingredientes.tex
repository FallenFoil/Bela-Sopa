% ---------------------------------------------------------------------------- %

\begin{table}[ht]
  \centering
  \tabelausecase
  \begin{tabularx}{\textwidth}{|>{\raggedright\let\newline\\\arraybackslash\hspace{0pt}}p{2.5cm}|>{\raggedright\let\newline\\\arraybackslash\hspace{0pt}}X|>{\raggedright\let\newline\\\arraybackslash\hspace{0pt}}X|}
    \hline
    \emph{Use case}: & \multicolumn{2}{l|}{Gerar lista de ingredientes} \\ \hline
    Pré-condição: & \multicolumn{2}{l|}{Estar autenticado} \\ \hline
    Pós-condição: & \multicolumn{2}{l|}{Lista de ingredientes guardada no sistema} \\ \hline
     & \textbf{Ator} & \textbf{Sistema} \\ \hline
    \multirow[t]{7}{=}{Comportamento Normal} & 1. Decide gerar a lista de ingredientes &  \\ \cline{2-3}
     &  & 2. Consulta ementa semanal do ator \\ \cline{2-3}
     &  & 3. Calcula os ingredientes necessários para a semana \\ \cline{2-3}
     &  & 4. Mostra os ingredientes e as respetivas quantidades \\ \cline{2-3}
     & 5. Confirma a lista de ingredientes. &  \\ \cline{2-3}
     &  & 6. Guarda a lista de ingredientes \\ \cline{2-3}
     & 7. <<extends>> aceder a serviços externos &  \\ \hline
    \multirow[t]{3}{=}{Comportamento Alternativo 1 [Ator quer adicionar ingrediente à lista] (passo 5)} & 5.1. Escolhe o ingrediente e quantidade a adicionar &  \\ \cline{2-3}
     &  & 5.2. Adiciona ingrediente à lista \\ \cline{2-3}
     &  & 5.3. Volta ao passo 5 \\ \hline
    \multirow[t]{3}{=}{Comportamento Alternativo 2 [Ator quer remover ingredientes à lista] (passo 5)} & 5.1. Escolhe o ingrediente e quantidade a remover &  \\ \cline{2-3}
     &  & 5.2. Remove ingrediente da lista \\ \cline{2-3}
     &  & 5.3. Volta ao passo 5 \\ \hline
    \multirow[t]{2}{=}{Comportamento Alternativo 3 [Quantidade do ingrediente é zero] (passo 5.2)} &  & 5.2.1 Indica que o ingrediente não existe na lista \\ \cline{2-3}
     &  & 5.2.2 Volta ao passo 5 \\ \hline
\end{tabularx}
  \caption{Especificação do \emph{use case} ``gerar lista de ingredientes''.}
  \label{tab:uc-gerar-lista-de-ingredientes}
\end{table}

% ---------------------------------------------------------------------------- %
